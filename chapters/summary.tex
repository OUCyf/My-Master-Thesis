\chapter{总结与展望}

本文主要对地震学的两个方面进行研究介绍,震源参数与背景噪声成像,这也是地震学研究的两个主要方向。
对震源的研究离不开对地下结构的支持,通常需要给定速度模型来定量分析地震震源参数。
对地下结构的研究也离不开对震源性质的分析,在利用天然地震进行成像时,需要我们分析地震发生的位置与震源机制;
在利用背景噪声成像时,也需要我们分析噪声源对经验格林函数的影响。震源与结构是紧密联系的。

在第一部分,我们详细回顾并推导了震源模型,并给出了研究大、中、小型地震的震源模型。
我们介绍了矩张量常用分解方式,并描述了两种矩张量可视化的方法,Hudson图与Lune图。
在基于上述震源理论知识,我们将贝叶斯蒙特卡洛的优化方法引入到震源参数反演中。
我们编写了MCMTpy的程序包,它能够有效地反演双力偶模型与矩张量模型参数,并提供了它们的后验概率。
我们将新方法应用到2021中国云南漾濞地震与2008美国Mt. Carmel地震的震源参数反演中,
并用Hudson图与Lune图可视化了矩张量反演结果的不确定性。
通过三十组随机初始模型,我们验证了该方法的鲁棒性。

在第二部分,我们详细回顾并推导了背景噪声干涉理论,并讨论了不同理论的假设存在的问题。
我们编写了大规模背景噪声互相关运算的程序,并将其应用到DAS阵列的实验中。
我们详细分析了2020年北京白家疃DAS观测实验中观测到的不同信号,并利用背景噪声信号进行了面波频散成像的研究。
我们发现仅使用基阶频散信号存在很大的局限性,高阶信号与全局优化方法在背景噪声成像的研究中非常重要。

在后面的工作中,有以下几个方法可以继续深入研究:
\begin{enumerate}
    \item 利用视震源时间函数的特点与高阶地震矩张量研究小地震的破裂方向性,并尝试研究小地震的破裂过程。
    \item 开发新的全局优化方法,进行背景噪声成像。
    \item 利用背景噪声监测地下速度的变化。
    \item 探讨在有限区域内,同时反演噪声源分布与格林函数的理论。
\end{enumerate}
